% This is a part of the paper presented at 
% the KyoungPook Ntnl U in Nov. 2004.
%
\documentclass[11pt]{article}
\usepackage{amsfonts,amsmath,amsthm,latexsym,amssymb}
\usepackage{graphicx}
%\usepackage{hfont}
%\usepackage[default,narrower]{hlatex-interword}

\newtheorem{theorem}{Theorem}
\newtheorem{definition}[theorem]{Definition}
\newtheorem{lemma}[theorem]{Lemma}
\newtheorem{corollary}[theorem]{Corollary}
\newtheorem{proposition}[theorem]{Proposition}

%\raggedbottom
\begin{document}
\title{Drawing a surface in $\mathbb{L}^3$}
\author{Young Wook Kim}
\date{Nov., 2004}

\maketitle

This paper is to introduce a known method of 
drawing minimal surfaces in $\mathbb{E}^3$ and show
how to use it to find and draw a family of maximal 
surfaces in the Minkowski space $\mathbb{L}^3$. 



\section{Maximal surfaces in $\mathbb{E}^3$}
The following table is a list of
maximal surfaces in $\boldsymbol{\mathbb{L}^3}$.


\renewcommand{\arraystretch}{2}
\begin{table}[h]
\begin{center}
\begin{tabular}{|l|l|c|c|}
\hline
    \textbf{Surfaces} & \textbf{Riemann Surface} $\boldsymbol{M}$ 
    & $\boldsymbol{f\,dz}$ & $\boldsymbol{g}$ \\
\hline\hline
    catenoid & $\mathbb{S}^2\smallsetminus\{ 0 , \infty \}$ 
    & $\dfrac{1}{z^2}\,dz$ & $z$ \\
\hline
    helicoid & $\mathbb{S}^2\smallsetminus\{ 0 , \infty \}$ 
    & $\dfrac{i}{z^2}\,dz$ & $z$ \\
\hline
    Enneper's surface & $\mathbb{C}$         & $dz$            & $z$ \\
\hline
    Trinoid  & $\mathbb{S}^2  
    \smallsetminus \{1, e^{2\pi i / 3}, e^{4\pi i / 3}  \}$
        & $\dfrac{1}{(z^3-1)^2}\,dz$ & $z^2$ \\
\hline
    Costa's surface & {?} & {?} 
    & {?}  \\
\hline
\end{tabular}
\end{center}
\caption{A table}
\end{table}




Minkowski space-time $\mathbb{L}^3$ is $\mathbb{R}^3=\{(x,y,t)\}$ with 
pseudo-riemannian metric $dx^2+dy^2-dt^2$.
Let $M$ be a Riemann surface, and $f,g: M \rightarrow \mathbb{C}$ be 
analytic functions. Then, the Weierstrass-type formula
\begin{equation*}
    Re \left\{ \int_{z_0}^z 
    \left( (1+g(w)^2) f(w), {\boldsymbol{i}}\,( 1-g(w)^2) f(w) , 
    2\,g(w)f(w) \right) dw \right\}
\end{equation*}
defines a {space-like} maximal immersion into $\mathbb{L}^3$.

For the periods of other components we have

\begin{lemma}\label{Lem:space-period}
    The period around $\gamma$ for the
    $xy$-components are 0 iff
    \begin{equation*}
        \sigma = \sqrt{\frac{1}{2} \frac{A}{B}}.
    \end{equation*}
\end{lemma}


\section{The Surface}
We render the surface graphics
using a 3D plotting function in Mathematica and we see the surface as above.
(Figure \ref{Afigure})%
\footnote{The figure number is not correct here.}
%\vspace{1cm}

\begin{figure}
\begin{center}\label{Afigure}
{\includegraphics[height=6cm]{./surf_max_CHM.eps}}
%%%%%%%%%%%%%%%%%%%%%%%%%%%%%%%%%Give the drawing of the surface
\end{center}
\caption{A figure}
\end{figure}



\section{Final remarks}
This gives a new surface which is in close relation with the
Costa-Hoffman-Meeks minimal surface\cite{hoffman_meeks}.


%%%%%%%%%%%%%%%%%%%%%%%%%%%%%%%%%%%%%%%

\begin{thebibliography}{9}

\bibitem{hoffman_meeks}
Hoffman, D.{} and W.{} H.{} Meeks, III, 
Embedded minimal surfaces of finite topology, \textsl{Ann.{} of Math.} 
\textbf{131} (1990), 1--34.


\end{thebibliography}

\end{document}


