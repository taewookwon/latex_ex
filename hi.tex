%!TEX program = xelatex
\documentclass{oblivoir}
    \usepackage{ikps,ansform}
    \usepackage{lipsum}
  
    \newcounter{problem}[section]
    \newenvironment{problem}{\noindent\refstepcounter{problem}\textbf{\large\theproblem.} }{}
    
    
    
\begin{document}
    
    드디어 라텍으로 문서를 처음 작성한다!
    
    \section{ES and MS}
    $A = \left\{ \dfrac{a}{b} \midbar \dfrac{a}{b}\text{는 자연수.} \right\}$
    
    $a\le b$ $a \ge b$ $a> b$ $a < b $ $a \ne b$
    
    $A$ $\mrm{A}$ $\mrm{A}_n$
    $\mrm{A_\mit n  B_\mit n}$
    
    $\ovr{AB}$  $\overline{\mathrm{AB}}$
    $\ovl{X}$ $\ovl{x}$
    
    $l \ppd m$, $l \ppd \alpha$,
    $l \prl m$, $l \plr \alpha$
    
    $\arc{AB}$ $\arc{AEB}$
    
    $\xy{a}{b}$ $\xy[P]{a}{b}$
    
    이건 ikps의 명령어가 아닙니다.
    $\triangle$ $\angle$ $\square$
    $\frac{a}{b}$ $\dfrac{a}{b}$
    $a^{\tfrac{n}{m}}$
    
    $\triangle \mrm{ABC} \equiv \triangle \mrm{DEF}$
    
    
    \section{High School Math I}
    $\abs{x}$ $\abs{\dfrac{a}{b}}$
    
    $\idp{2}{3}{1}{2}$, $\IDP{a}{b}{m}{n}$, $\dfrac{mb + na}{m + n}$
    
    점 $\xy[A]{1}{2}$와 $\xy[B]{4}{5}$를 $1:2$로 내분하는 점 $\mrm M$의 좌표는
    $\xy[M]{\idp{1}{4}{1}{2}}{\idp{2}{4}{1}{2}}$이다.
    
    $\edp{1}{2}{3}{4}$, $\EDP{a}{b}{m}{n}$
    
    $\tidp{a}{b}{m}{n}$, $\tedp{a}{b}{m}{n}$
    $\TIDP{a}{b}{m}{n}$, $\TEDP{a}{b}{m}{n}$
    
    \section{High School Math II}
    
    before $A \cap B$, $A \cup B$, $U^C = \varnothing =\emptyset$
    
    after
    
    $p$, $\sim q$,$p \longrightarrow q$ $p, \longleftarrow q$ $p$ $p \Longrightarrow q$ $p \Longleftrightarrow q$
    
    $\comp{f}{g}$, $\COMP{f}{g}{x}$, $f^{-1}$
    
    $a^b$, $a^{\tfrac{n}{m}}$, $a\expo{\tfrac{n}{m}}$
    
    $\log_{a^m} {b^n} = \dfrac{n}{m} \log_a b$, $\sum_{k=1}^n f(n) = \dfrac{n(n+1)}{2}$
    
    \section{Calculus I}
    
    $\lim_{n \to \infty} \dfrac{2^{n-1} + 3}{2^n + 1}$ $\lim_{x \to a} f(x)$ 
    
    $\xy{a}{b}$
    
    $f' (f+g)' f''$
    
    $\dfrac{f(b)-f(a)}{b-a} = f'(c)$
    
    기본 $\int_{2a} ^{4b} f(x)dx $
    
    재정의 $\intg{2a}{4b}{f(x)dx}$
    
    $\dfrac{d}{dx}\intg{a}{x}{f(t)dt} = f(x)$
    
    $\intg{a}{b}{f(x)dx} = \bigg[ F(x) \bigg]_{a}^{b}$
    
    \section{Calculus II}
    $\sin x$ $\cos x$ $\tan x$ $\cot x$ $\sec x$ $\csc x$ $a^x$ $e^x$ $\log_a x$ $\ln x$
    
    $\intg{a}{b}{f(g(x))g'(x)dx} = 
    \intg{g(a)}{g(b)}f(t)dt$
    
    $\intg{a}{b}{f(x)g'(x)dx} = \bigg[ f(x)g(x) \bigg]_a ^b - \intg{a}{b}{f'(x)g(x)dx}$
    
    \section{Geometry and Vector}
    $\vec{a}$ $\vrm{AB}$ $\avi{a}$ $\avr{AB}$
    $\avr{P'Q'}$
    
    $\vec{a} = \xy{a}{b}$
    $\vrm{PQ} = \xyz{a}{b}{c}$
    $\xy[M]{p}{q}$
    $\xyz[N]{p}{q}{r}$
    
    $\vrm{AB} \bcd \vrm{PQ}$
    
    \section{Probability and Statistics}
    $_n\mrm C _r = _n\mrm C _r $
    
    $\PERM{n}{r}$, $\COMB{n}{r}$, $\HOMO{n}{r}$, $\PROD{n}{r}$, $\dfrac{n!}{a!b!}$, $\SNK{n}{k}$, $\PNK{n}{k}$
    
    $\mrm{P}\left(A\right)$
    
    $\pr{A}$
    $\pr{A \cap B}$ $\pn AB$
    $\pr{A \cup B}$ $\pu A{B^C}$
    $A \cap B$, $A \cup B$, $U^C = \varnothing =\emptyset$
    $\E{X}$ $\E{X^2}$  $\V{X}$ $\SIG{X}$
    $\PR{X=1}$ $\pr{X=1}$
    
    
    $\ND{m}{\sigma^2}$ $\BD{n}{\dfrac{1}{3}}$
    
    $\sigma \left( \dfrac{a}{b}\right)$ $\SIG{\dfrac{a}{b}}$
    
    \newpage
    
    $a_1, a_2, \cdots, a_n \ldots \dots$ 


      
    \begin{problem}
        양수 $t$에 대하여 구간 $\COI{1}{\infty}$에서 정의된 함수 $f(x)$가
        \[ f(x) = 
        \begin{cases}
        \ln x & (1 \le x < e) \\
        -t + \ln x & (x \ge e)
        \end{cases}
        \]
        일 때, 다음 조건을 만족시키는 일차함수 $g(x)$ 중에서 직선 $y=g(x)$의 기울기의 최솟값을 $h(t)$라 하자.
        \begin{justbox}
        $1$ 이상의 모든 실수 $x$에 대하여 $(x-e)\{ g(x) - f(x) \} \ge 0$이다.
        \end{justbox}
        미분가능한 함수 $h(t)$에 대하여 양수 $a$가 $h(a) = \dfrac{1}{e+2}$을 만족시킨다. $h'\left(\dfrac{1}{2e}\right) \times h'(a)$의 값은? [4점]
        \anstwo{2}{4}{6}{8}{10}
    
        \end{problem}
        %\ansone{2}{4}{6}{8}{10}
        %\ansthree{2}{4}{6}{8}{10}
        %\ansfive{2}{4}{6}{8}{10}
        
    
        \begin{problem}
        그림과 같이 두 초점이 $\mrm F$, $\mrm F'$인 쌍곡선 $\dfrac{x^2}{8} - \dfrac{y^2}{17} = 1$ 위의 점 $\mrm P$에 대하여 직선 $\mrm{FP}$와 직선 $\mrm{F'P}$에 동시에 접하고 중심이 $y$축 위에 있는 원 $C$가 있다. 직선 $\ovr{F'P}$와 원 $C$의 접점 $\mrm Q$에 대하여 $\ovr{F'Q}=5\sqrt2$일 때, $\ovr{FP}^2 + \ovr{F'P}^2$의 값을 구하시오. (단, $\ovr{F'P}<\ovr{FP}$) [4점]
        
        \begin{figure}[h!]
            \centering
            \includegraphics[scale=0.5]{a.png}
            \caption{문제}
        \end{figure}
        \end{problem}
    
        \begin{problem}
        그림과 같이 두 초점이 $\mrm{F}$, $\mrm{F'}$인 쌍곡선 $\dfrac{x^2}{8}-\dfrac{y^2}{17}=1$ 위의 점 $\mrm{P}$에 대하여 직선 $\mrm{FP}$와 직선 $\mrm{F'P}$에 동시에 접하고 중심이 $y$축 위에 있는 원 $C$가 있다. 직선 $\ovr{F'P}$와 원 $C$의 접점 $\mrm{Q}$에 대하여 $\ovr{F'Q}=5\sqrt{2}$일 때, $\ovr{FP}^2 + \ovr{F'P}^2$의 값을 구하시오. (단, $\ovr{F'P}<\ovr{FP}$) [4점]
   \end{problem} 

   \begin{condition}{(가)}
   \item 허혁재는 목소리가 징그럽다.
   \item 크레타 섬의 모든 사람들은 거짓말쟁이이다. 
   \item 나는 크레타 섬의 사람이다.
   \end{condition}

   \begin{condition}{a.}
    \item 허혁재는 목소리가 징그럽다.
    \item 크레타 섬의 모든 사람들은 거짓말쟁이이다. 
    \item 나는 크레타 섬의 사람이다.
    \end{condition}
 
\begin{selection}{ㄱ.}
    \item 허혁재는 목소리가 징그럽다.
    \item 크레타 섬의 모든 사람들은 거짓말쟁이이다. 
    \item 나는 크레타 섬의 사람이다.
\end{selection}

\begin{selection}{a.}
    \item 허혁재는 목소리가 징그럽다.
    \item 크레타 섬의 모든 사람들은 거짓말쟁이이다. 
    \item 나는 크레타 섬의 사람이다.
\end{selection}


\ansganada{a,b,c}{a,b,c}{b,c,d}{c,d,e}{d,e,f}

\ansgana{a,b}{a,b}{b,c}{c,d}{d,e}

    \end{document}
      